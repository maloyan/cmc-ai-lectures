%----------------------------------------------------------------------------------------
%	PACKAGES AND THEMES
%----------------------------------------------------------------------------------------
\documentclass[aspectratio=169,xcolor=dvipsnames]{beamer}
\usetheme{Simple}

\usepackage{hyperref}
\usepackage{graphicx} % Allows including images
\usepackage{booktabs} % Allows the use of \toprule, \midrule and \bottomrule in tables
\usepackage{graphicx}
\usepackage{amsmath}

\usepackage{mathtools}% superior to amsmath
\usepackage{tikz}
\usepackage{animate}

\makeatletter
\newcommand\mathcircled[1]{%
  \mathpalette\@mathcircled{#1}%
}
\newcommand\@mathcircled[2]{%
  \tikz[baseline=(math.base)] \node[draw,circle,inner sep=15pt] (math) {$\m@th#1#2$};%
}
\makeatother
%----------------------------------------------------------------------------------------
%	TITLE PAGE
%----------------------------------------------------------------------------------------

% The title
\title[short title]{Simple Beamer Theme}
\subtitle{Lecture 13}

\author[Narek Maloyan] {Narek Maloyan}
\institute[NTU] % Your institution may be shorthand to save space
{
    % Your institution for the title page
    Faculty of Computational Mathematics and Cybernetics \\
    Lomonosov Moscow State University
    \vskip 3pt
}
\date{\today} % Date, can be changed to a custom date


%----------------------------------------------------------------------------------------
%	PRESENTATION SLIDES
%----------------------------------------------------------------------------------------

\begin{document}

\begin{frame}
    % Print the title page as the first slide
    \titlepage
\end{frame}

\begin{frame}{Overview}
    % Throughout your presentation, if you choose to use \section{} and \subsection{} commands, these will automatically be printed on this slide as an overview of your presentation
    \tableofcontents
\end{frame}

%------------------------------------------------
\section{What is neuron?}
%------------------------------------------------

\begin{frame}{Neuron}
    $$\mathcircled{0.4}$$
    \begin{center}
        \textbf{Neuron} $\rightarrow$ Thing that holds a number       
    \end{center}
\end{frame}

\begin{frame}{Activation map}
    \begin{center}
        \includegraphics[width=0.8\textheight]{../images/mnist_cnn_keras_8_0.png}
    \end{center}
\end{frame}

\begin{frame}{Forward pass}
    \begin{center}
        \animategraphics[loop,controls,width=\textheight]{10}{../images/nn_animation-}{0}{107}
    \end{center}
\end{frame}

%------------------------------------------------
\section{What is intelligence?}
%------------------------------------------------
\begin{frame}{Patterns}
    \begin{center}
        \includegraphics[width=\textheight]{../images/patterns.png}
    \end{center}
\end{frame}

%------------------------------------------------
\section{Neural network}
%------------------------------------------------
\begin{frame}{General formula}
    $$f = \sigma (Wx + b)$$
\end{frame}
%------------------------------------------------

%------------------------------------------------
\section{Universal approximation theorem}
%------------------------------------------------
\begin{frame}{Universal approximation theorem}
    \begin{theorem}[Mass--energy equivalence]
       2-layer NNs with sigmoid activation function can approximate any other function
    \end{theorem}
\end{frame}
%------------------------------------------------

\begin{frame}{Multiple Columns}
    \begin{columns}[c] % The "c" option specifies centered vertical alignment while the "t" option is used for top vertical alignment

        \column{.45\textwidth} % Left column and width
        \textbf{Heading}
        \begin{enumerate}
            \item Statement
            \item Explanation
            \item Example
        \end{enumerate}

        \column{.5\textwidth} % Right column and width
        Lorem ipsum dolor sit amet, consectetur adipiscing elit. Integer lectus nisl, ultricies in feugiat rutrum, porttitor sit amet augue. Aliquam ut tortor mauris. Sed volutpat ante purus, quis accumsan dolor.

    \end{columns}
\end{frame}

%------------------------------------------------
\section{Second Section}
%------------------------------------------------

\begin{frame}{Table}
    \begin{table}
        \begin{tabular}{l l l}
            \toprule
            \textbf{Treatments} & \textbf{Response 1} & \textbf{Response 2} \\
            \midrule
            Treatment 1         & 0.0003262           & 0.562               \\
            Treatment 2         & 0.0015681           & 0.910               \\
            Treatment 3         & 0.0009271           & 0.296               \\
            \bottomrule
        \end{tabular}
        \caption{Table caption}
    \end{table}
\end{frame}

%------------------------------------------------

\begin{frame}{Theorem}
    \begin{theorem}[Mass--energy equivalence]
        $E = mc^2$
    \end{theorem}
\end{frame}

%------------------------------------------------

\begin{frame}{Figure}
    Uncomment the code on this slide to include your own image from the same directory as the template .TeX file.
    %\begin{figure}
    %\includegraphics[width=0.8\linewidth]{test}
    %\end{figure}
\end{frame}

%------------------------------------------------

\begin{frame}[fragile] % Need to use the fragile option when verbatim is used in the slide
    \frametitle{Citation}
    An example of the \verb|\cite| command to cite within the presentation:\\~

    This statement requires citation \cite{p1}.
\end{frame}

%------------------------------------------------

\begin{frame}{References}
    % Beamer does not support BibTeX so references must be inserted manually as below
    \footnotesize{
        \begin{thebibliography}{99}
            \bibitem[Smith, 2012]{p1} John Smith (2012)
            \newblock Title of the publication
            \newblock \emph{Journal Name} 12(3), 45 -- 678.
        \end{thebibliography}
    }
\end{frame}

%------------------------------------------------

\begin{frame}
    \Huge{\centerline{The End}}
\end{frame}

%----------------------------------------------------------------------------------------

\end{document}